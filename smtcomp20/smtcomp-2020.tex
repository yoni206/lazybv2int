\documentclass{easychair}
\usepackage[utf8]{inputenc}
\usepackage{authblk}
\usepackage{cite}
\usepackage{authblk}
\usepackage{hyperref}
\usepackage{xcolor}

\newcommand{lazybvtoint}{lazybv2int}
\newcommand{smtcomp}{SMT-COMP}
\newcommand{smtlib}{SMT-LIB}
\newcommand{msat}{mathsat}
\newcommand{cvcfour}{CVC4}

\begin{document}

\author{
		Ahmed Irfan\inst{1} \and
		Makai Mann\inst{1} \and
		Andres N\"otzli\inst{1} \and
		Andrew Reynolds\inst{2}\and
		Yoni Zohar\inst{1}
		Clark Barrett\inst{1}
}
\institute{
  Stanford University \and
  The University of Iowa
 }

\title{LazyBV2Int at the SMT Competition 2020}

\maketitle


\noindent
\begin{abstract}
TBD
\end{abstract}

\section*{Overview}

\lazybvtoint is a prototype SMT-solver for the theory of fixed-width bit-vectors.
It also supports uninterpreted functions, however, it will only compete in QF\_BV tracks.
The basic idea behind the tool is a 
translation of bit-vector formulas into formulas over integer 
arithmetic and uninterpreted functions.

\section*{Wrapper Tool}
According to the definitions of various types of solvers published by the organizers
of \smtcomp 2020, \lazybvtoint is a {\em wrapper tool} (see \cite{}).
To parse the input problem (in \smtlib format), \lazybvtoint employs \msat's parser through an API \cite{};
And to solve the translated problem it uses \cvcfour \cite{}.
The interface to both external solvers, as well as many parts of the implementation,
use \smtswitch \cite{}.

\section*{Technical Details}

\section*{Configurations}

\newpage% align right and left column in size

%  \bibliography{smtcomp-2020}
%  \bibliographystyle{plain}

\end{document}
